\documentclass[12pt,a4paper]{report}
\synctex=1
\usepackage[utf8]{inputenc}
\usepackage[margin=2cm, top=1.5cm, bottom=2cm]{geometry}
\usepackage{graphicx}
\usepackage{libertine}
\usepackage{amsmath}
\usepackage{amssymb}
\usepackage{listings}
\usepackage{pgfornament}
\usepackage{eso-pic}
\usepackage{textcomp}
\usepackage{courier}
\usepackage[hangul]{kotex}
\usepackage{rotating}
\usepackage{dirtree}
\title{
	\centering
	\pgfornament[width=12cm,color=teal]{84}\\
	\vspace{1cm}
	\fontsize{50}{50} \selectfont {오목 인공지능 형성\\설계과제 최종 보고서}\\
		\pgfornament[width=12cm,color=teal]{88}\\
	\vfill}
\author{
	\LARGE
	\begin{tabular}{rl}
		\hline
		교과목명 : & 자료구조와 실습\\
		담당교수 : & 정 준호 교수님\\
		프로젝트명 : & $ \Omega $目\\
		학번 : & 2016110056\\ 
		이름 : & 박승원\\
		날짜 : & \today\\
		\hline
	\end{tabular}\vspace{1cm}
	\\
\includegraphics[width=0.5\textwidth]{logo.jpg}
	}
\date{}


\linespread{1.3}

\begin{document}

\maketitle

%\includegrap

\newpage
\tableofcontents
\newpage
\chapter*{과제 요약서}
\paragraph{설계 과제명} 학습형 오목 인공지능 형성\\ 
\paragraph{주요기술용어} 인공지능, Machine Learning, 오목, 해싱, 이진트리\\
\paragraph{1. 과제목표} 오목 게임의 기초적인 룰만을 바탕으로 컴퓨터끼리의 대국을 통하여 실력을 늘려나가는 학습형 인공지능을 형성한다.\\
\paragraph{2. 수행 내용 및 방법} C를 이용하여 Makefile 프로젝트로  프로그램 작성\\
\paragraph{3. 수행 결과} 우리 \\
\paragraph{4. 결과 분석} 어느 정도 사람과 두어볼 수 있는 오목 인공지능이 형성되었다.\\

\noindent
\chapter{서론}
\section{설계과제 목적}
\begin{itemize}
\item 창의적 사고 및 다양한 방법을 통한 문제 해결 능력
\item 여러 가지 제약조건을 고려한 효율적인 프로그램 설계 능력
\end{itemize}
\section{설계과제 내용}
\section{진행일정 및 개인별 담당 분야}
\chapter{프로그램의 구조 및 구성}
\section{전체 구성도}


\section{프로그램 세부 구성}
\subsection{Tree 구조와 파일}

\dirtree {%
.1 root.
.3 Makefile : 하위 디렉토리의 Makefile들을 실행시킴.
.3 serverip.cfg : 클라이언트가 서버의 IP주소를 찾을 파일.
.3 facility.txt : 자신의 업체의 시설 상황 파일.
.2 src. 
.3 Makefile.
.3 server.cpp.
.3 client.cpp.
.3 console\_front.cpp.
.3 tcpip.cc : TCPIP 모듈.
.3 tcpip.h.
.3 reserv.cc : wrapper class.
.3 reserv.h.
.3 reser.cc : C언어, List를 사용한 자료구조와 함수.
.2 gtk.
.3 Makefile.
.3 frontend.cc.
.3 frontend.cpp.
.3 frontend.h.
.2 OBJ.
.3 Makefile : obj파일들을 링크하는 역할.	
}


\chapter{결과 및 토의}
\section{프로그램 테스트 결과}
\section{수행 결과에 대한 토의}
\section{기타}
\chapter{부록}
매뉴얼 : 별첨\\ 소스코드 : 별첨
\end{document}
